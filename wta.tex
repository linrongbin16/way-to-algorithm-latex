%-*- coding: UTF-8 -*-
% gougu.tex
% 勾股定理
\documentclass[UTF8]{ctexart}
\title{算法之路}
\author{林荣彬}
\date{\today}
\bibliographystyle{plain}
\newtheorem{thm}{定理}

\begin{document}

\maketitle
\tableofcontents
\section{勾股定理在古代}
\section{勾股定理的近代形式}
\bibliography{math}

西方称勾股定理为毕达哥拉斯定理,将勾股定理的发现归功于公元前6世纪的毕达哥拉斯学派。该学派得到了一个法则,可以求出可排成直角三角形三边的三 元数组。毕达哥拉斯学派没有书面著作,该定理的严格表述和证明则见于欧几里 德《几何原本》的命题␣47:``直角三角形斜边上的正方形等于两直角边上的两个正方形之和。''证明是用面积做的。

我国《周髀算经》载商高(约公元前␣12␣世纪)答周公问:
\begin{quote}
    勾三股四弦五
\end{quote}

\begin{thm}[勾股定理]
    直角三角形斜边的平方等于两腰的平方和
\end{thm}

\end{document}

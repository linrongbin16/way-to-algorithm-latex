%-*- coding: UTF-8 -*-
% gougu.tex
% 勾股定理

\documentclass[UTF8]{ctexart}
\usepackage{graphicx}
\usepackage[colorlinks=false]{hyperref}
\usepackage{float}

\title{算法之路}
\author{林荣彬}
\date{\today}
\bibliographystyle{plain}
\newtheorem{theorem}{定理}

\begin{document}

\maketitle
\tableofcontents
\section{勾股定理在古代}
\section{勾股定理的近代形式}
\bibliography{math}

西方称勾股定理为毕达哥拉斯定理,将勾股定理的发现归功于公元前6世纪的毕达哥拉斯学派。该学派得到了一个法则,可以求出可排成直角三角形三边的三 元数组。毕达哥拉斯学派没有书面著作,该定理的严格表述和证明则见于欧几里 德《几何原本》的命题␣47:``直角三角形斜边上的正方形等于两直角边上的两个正方形之和。''证明是用面积做的。

我国《周髀算经》载商高(约公元前␣12␣世纪)答周公问:
\begin{quote}
    勾三股四弦五
\end{quote}

\begin{theorem}[勾股定理]
    直角三角形斜边的平方等于其他两边平方和
\end{theorem}

\begin{figure}[ht]
    \centering
    \includegraphics[scale=0.8]{wta-image/cover-keyboard.jpg}
    \caption{封面示例截图}
    \label{fig:xiantu}
\end{figure}

\begin{table}[H]
    \begin{tabular}{|ccc|}
        \hline
        直角边 $a$  &  直角边 $b$  &  斜边 $c$  \\
        \hline
        3 & 4 & 5   \\
        5 & 12 & 13 \\
        \hline
    \end{tabular}
\end{table}

\end{document}
